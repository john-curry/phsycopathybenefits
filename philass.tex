\documentclass[12pt,jou]{apa}
\usepackage{times}
\usepackage[margin=1.0in]{geometry}
\usepackage{apacite}
\bibliographystyle{apacite}
\usepackage{setspace}
\usepackage{fancyhdr}
\doublespacing
\setlength{\parindent}{1cm}
\begin{document}
\begin{titlepage}
\textsc{\small Running Head: Psychopathy Examination}\\
\begin{center}
\textsc{\huge Psychopathy and it's Effect On Society}\\[1.5cm]

\begin{minipage}{0.4\textwidth}
\begin{flushleft} \large
John \textsc{Curry}\\
Camosun Collage
\end{flushleft}
\end{minipage}
\vfill
{\large \today}
\end{center}
\end{titlepage}


\pagenumbering{gobble}
\tableofcontents                  
\newpage
\pagenumbering{arabic}
\pagestyle{fancy}
\fancyhf{}
\fancyhead[R]{\thepage}
\fancyhead[L]{Psychology Examination}

\section{Introduction}
Psychopathy is a problem that is prevalent in modern day society. Psychopathic individuals have a tendency towards certain personality traits that can cause harm to the people around them and can harm society as a whole. With personality traits like deception, irresponsibility, lack of forward planning, impulsivity, lack of empathy, lack of guilt, antisocial behaviour, and stimulation seeking behaviour ~\cite{brazil} it is not hard to see the effects a psychopath could have in any number of situation. Take, for example, a psychopath who is in need of money. This individual, when taking into account their irresponsibility and impulsivity,  might walk into a bank and simply rob it for some extra cash. Indeed it is hard for someone without psychopathic tendencies to even conceive this idea, but to a psychopath it has no more weight to it then picking up a twenty dollar bill sitting on the side of the road. 

Although it may be easy to see how psychopaths have an effect on society as a whole, this report will examine psychopathy and it will show how psychopath have an effect on many areas of society. 

\section{Psychopaths and The Criminal Justice System}
Psychopaths have a disproportionate representation in the criminal justice system. The rate by which psychopaths commit crimes and end up incarcerated it 15 to 25 times more likely then the average non-psychopath \cite{crimpsych}. With numbers like theses it is clear that even just having psychopaths in prison, they end up draining society through court proceedings, police time, and incarceration costs. Although it has been shown that psychopaths have a higher rate of conditional release then non-psychopaths \cite{crimpsych} this is countered by their willingness to lie and deceive to get what they want. Psychopaths also have a higher recidivism than people who are not psychopaths with a recidivism rate as high as 80\% \cite{crimpsych}. Because psychopaths lack the ability to realize that what they are doing is wrong, they usually end up going back to their life of crime.   

Of the prison population, psychopaths take up a whopping 20\%, which is astonishing since psychopaths only take up 1\% of the general population \cite{crimpsych}. This shows that psychopaths have a major impact on the criminal justice system just by sheer numbers alone, but when looking at the cost, the cost of psychopathy is almost unimaginable. The cost for psychopaths when taking into consideration police, public defenders, property damage, jurors, prosecutors, property loss, and jail time, is approximately \$460 billion dollars to the American Criminal Justice system \cite{crimpsych}. 

Indeed it is also hard to properly figure out if a psychopath is responsible for their actions. One could argue that because a psychopath is perfectly rational and that they knew exactly what they were doing, the psychopath should be punished to the full extent of the law, but there is a flip side to this. There is growing evidence that psychopaths have a neurological problem \cite{crimpsych} and therefore they should not punished. But what to do with them instead of punishment and imprisonment? 


\section{Corporate Psychopaths}

Not only do psychopaths end up in the prison system, but there is growing evidence that psychopaths have made their way into the workforce. A corporate psychopath is a psychopath that works in an organization \cite{corptheory}. These psychopaths manipulate people in order to attain their own goals. Corporate psychopaths have poor organizational management skills and tend to bully, lie, and cheat their way to corporate success. These results can have huge implications on the people that work for these individuals, but also on society as a whole.

It has been postulated that the Global Financial Crisis of 2007 - 2008 may have been caused by Corporate Psychopaths \cite{corptheory}. Corporate Psychopaths have been found to be more likely to be in senior positions in large financial institutions \cite<as cited in>{corptheory}. If indeed this is true, it has large implications for the impact of psychopaths. People loose jobs, life saving, and houses when corporations fall. 

It is also easy to see how psychopaths could have effected this crisis. Psychopaths are risk takers by nature and may have played hard and fast with corporate investments. Psychopaths are also willing to lie and cheat in order to gain what they need, so they may use questionable trading and business practices in order to attain their goals. Also, they have no moral consideration in their decision making process, so they are willing to cause damage to others in order to accomplish their goals. 

Psychopaths are likely to rise to management positions quickly since they have charisma, charm, and confidence, so it is not surprising that psychopaths have been found in management \cite{ruthlesspsych}. It was indeed shown that in a sample of white collar workers, that 32.1\% had worked for a person with psychopathic tendencies, and that 5.75\% of people currently work for one \cite{ruthlesspsych}. It could postulate that because psychopaths have little regard for others, they may bully fellow employees in order to accomplish their goals. It was shown in a British study that psychopaths account for 35.2\% of bullying in the work place \cite{managepsych}. 

This shows that Corporate Psychopaths have an impact on the organizations they are working for, and anyone working in these organization might be working for a psychopath. 

Since psychopaths have little concern for anyone besides themselves, it would be logical to follow that they would have little interest on corporate social responsibility. This would have large implications on society as a whole in areas such as public safety, environmental protection, and political relations. In a study on corporate psychopaths and corporate social responsibility, it was found that as the level corporate psychopathy increased, corporate responsibility decreased. \cite{organizepsych}

\section{Psychopathy in the Community}

As it happens, a large portions of psychopaths do not end up in prison. It is estimated that up to 2 million psychopaths live in North America \cite<as cited in>{commpsych}. These individuals would have a detrimental effect on the community given their narcissistic and machiavellian behaviour, but might go unnoticed because of their charm, family connections, and smarts. 

In a study done on psychopaths in a community sample, psychopathic behaviour was found to be linked to negative community behaviour. Psychopathic characteristics were correlated with violent sentiments and was more pronounced when participants ranked high in callous-affect and impulsive manipulation. Psychopathy was also linked with alcohol abuse. It was also shown that psychopaths were more likely to use violence in provoking situations, as well as in situation where violence could be used in order to achieve their goal. \cite{commpsych}

These results have wide range of implications on the community. Violent behaviour could lead to harm to people living around these individuals. People working or living with these psychopaths might adopt this defective behavior when seeing how it can be used to accomplish certain goals. People might also end up feeling poorly about their living conditions when they feel they are being manipulated or abused by these individuals.

\section{Psychopathy and Treatment}

Psychopathy is highly regarded as a condition that is not only hard to fix, but that it may be impossible. When psychopaths went into group therapy, it was shown that they had a higher recidivism rate \cite<as cite in>{crimpsych}. Typically people in these types of groups were talking about their emotions, what makes them feel guilty, shameful, etc, but psychopaths do not feel these emotions. Indeed they may even use these groups to learn how to manipulate people. 

There is a concern with how to treat a psychopath. Psychopaths have been shown to be extremely impervious to classical rehabilitation techniques and in fact they can be detrimental to the psychopath \cite{crimpsych}. Techniques, such as psychotherapy, are aimed at patience who know that they have a problem and they want to be fixed. Unfortunately, not only do psychopaths not want to be fixed, but they believe that they do not have a problem. And again, their willingness to lie and deceive means that one on one therapy will most likely end in poor results. These results beg the question on what to do with psychopaths once they are in the system.

The evidence seems pretty dire that treatment may be useless on psychopaths but there may be some hope. A treatment called decompression treatment, whereby psychopathic individuals were put in long-term, one on one therapy, has been shown to reduce recidivism from 20\% to 10\% \cite<as cited in>{crimpsych}. These results are encouraging and may show some hope in effectively treating psychopaths.
\section{Conclusion}

Psychopath is a major problem in modern society. It has a wide range of implications and, unfortunately there is very little treatment for these individuals. Psychopaths have direct impact on our daily lives, through work or in our community and can lead to a decreased quality of life through acts such as bullying. And the big picture is even scarier. If psychopathy is indeed responsible for the collapse of the global economy, then it will be impossible to ignore that psychopathy has a major impact on society. Psychopathy is also a massive drain of resources when considering the major representation that they have in the criminal justice system.
\newpage
\bibliography{crimass}
\end{document}
