\documentclass[12pt,jou]{apa}
\usepackage{times}
\usepackage[margin=1.0in]{geometry}
\usepackage{apacite}
\bibliographystyle{apacite}
\usepackage{setspace}
\usepackage{fancyhdr}
\doublespacing
\setlength{\parindent}{1cm}
\begin{document}
\begin{titlepage}
\textsc{\small Running Head: Psychopathy Examination}\\
\begin{center}
\textsc{\huge Psychopathy and it's Effect On Society}\\[1.5cm]

\begin{minipage}{0.4\textwidth}
\begin{flushleft} \large
John \textsc{Curry}\\
Camosun Collage
\end{flushleft}
\end{minipage}
\vfill
{\large \today}
\end{center}
\end{titlepage}


\pagenumbering{gobble}
\tableofcontents                  
\newpage
\pagenumbering{arabic}
\pagestyle{fancy}
\fancyhf{}
\fancyhead[R]{\thepage}
\fancyhead[L]{Psychopathy Benefits}

\section{Introduction}
Psychopaths are individuals who have a certain set of personality disorders. When dealing with a certain psychological disorder, the nomenclature is a little unfortunate, and for this paper the term to describe psychopaths will be personality traits. The specific personality traits that these individuals tend to exhibit include deception, irresponsibility, lack of forward planning, impulsivity, lack of empathy, lack of guilt, antisocial behaviour, and stimulation seeking behaviour ~\cite{brazil}. It is not hard to see the effects a psychopath could have in any number of situation. When diagnosing psychopathic individuals, the most used and widely respected measure for a persons psychopathy is the \textit{Have Psychopathy Checklist - Revised }(PCL-R)\cite{nickerson2014}. The PCL-R measures a wide range of personality traits and ranks them on a scale of 0 to 2, adds them up, and gives the person a score out of 40. Anyone over a score of 30 is labeled a psychopath. The specific traits are as follows "glib and superficial charm, exaggerated grandiosity, need for stimulation, pathological lying, cunning and manipulativeness, lack of remorse or guilt; shallow affect (superficial emotional responsiveness), callousness and lack of empathy, parasitic lifestyle, poor behavioral controls, sexual promiscuity, early behavior problems, lack of realistic long-term goals, impulsivity, failure to accept responsibility for own actions, many short-term marital relationships, juvenile delinquency, revocation of conditional release, and criminal versatility” \cite{hareharpur1991}. Don't be surprised if you can think of a few people one can fit a fair amount of theses criteria; the psychopath is a large sum of all these traits and not just a subset of them. These traits tend to lead an individual with them to a life of criminality and imprisonment. 

The research done on psychopaths tends to be done of incarcerated individuals who have managed to get themselves prison time because of their ailment. This has led to a bias in the research data. Because psychopaths have a tendency to get themselves into prison, research has been focused to the psychopaths in prison and diagnostic tools used to detect psychopaths were developed to detect psychopathy in prisoners. The research is slightly biased towards individuals who have a criminal history. Although a psychopath is more likely to go to prison then the non-psychopathic individual, there is a large about of psychopaths, labeled Successful Psychopaths, that have managed to successfully navigate away from the criminal justice system.  
\newpage

\bibliography{crimass}
\end{document}
